% Inclusione dei pacchetti
\usepackage{mathtools}
\usepackage{cancel}
\usepackage{stmaryrd}
\usepackage{tikz}
\usepackage{pgfplots}

\usepgfplotslibrary{fillbetween}

% Definizione di nuove macro
\newcommand{\limplies}  {\rightarrow}
\newcommand{\valu}[1]   {\llbracket\;{#1}\;\rrbracket}
\newcommand{\taut}      {\;\vDash\;}
\newcommand{\repl}[2]   {[\;{#1}\,/\,{#2}\;]}
\newcommand{\liff}      {\longleftrightarrow}
\newcommand{\tc}        {\;\text{ t.c. }\;}

\pgfplotsset{
    functions/.style = {
        % Griglia
        grid             = both,
        minor grid style = { lightgray!10 },
        major grid style = { lightgray!50 },
        % Assi cartesiani
        axis lines     = middle,
        xlabel         = { $x$ },
        ylabel         = { $y$ },
        minor tick num = 5,
        % Funzione
        samples = 100,
        domain  = -8.5 : 8.5,
        xmin = -8.5, xmax = 8.5,
        ymin = -8.5, ymax = 8.5,
        % Dimensioni
        width = 13cm, height = 8.5cm
    }
}


% \addplot[domain = 0:5.5, blue, thick] { 1/x }
% node[at = {(axis cs:4,0.5)}] { $f(x)=\frac{1}{x}$ };
% \addplot[violet, thick]
% coordinates { (1,1) (0,0) (2,1/2) };
% \addplot[violet, thick]
% coordinates { (1,1) (1,0) };
% \addplot[violet, thick]
% coordinates { (2,1/2) (2,0) };
% \addplot[violet, mark = *]
% coordinates { (0.8,0.1) }
% node[pin = 240: { $A_2$ }] {};
% \addplot[violet, mark = *]
% coordinates { (1.3,0.1) }
% node[pin = 300: { $A_4$ }] {};
% \addplot[violet, mark = *]
% coordinates { (0.8,0.4) }
% node[pin = 120: { $A_1$ }] {};
% \addplot[violet, mark = *]
% coordinates { (1.3,0.4) }
% node[pin = 70: { $A_3$ }] {};